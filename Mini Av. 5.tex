\documentclass[a4paper,12pt]{article}
\usepackage[brazil]{babel}
\usepackage[utf8]{inputenc}
\usepackage{amsmath}
\usepackage{listings}
\usepackage{color}
\usepackage{hyperref}
\usepackage{graphicx}
\usepackage{amssymb}


\definecolor{codegray}{rgb}{0.5,0.5,0.5}
\definecolor{codeblue}{rgb}{0.3,0.3,0.9}
\definecolor{codegreen}{rgb}{0,0.6,0}

\lstset{
    language=R,
    basicstyle=\ttfamily\small,
    keywordstyle=\color{codeblue},
    stringstyle=\color{codegreen},
    commentstyle=\color{codegray},
    frame=single,
    numbers=left,
    numberstyle=\tiny\color{codegray},
    stepnumber=1,
    numbersep=5pt,
    showstringspaces=false,
    breaklines=true,
    breakatwhitespace=true,
    tabsize=4,
    captionpos=b
}

\title{SME0821 - Mini Avaliação 5}
\author{Ada Maris Pereira Mário}
\date{\today}

\begin{document}

\maketitle

\tableofcontents

\section{Introdução}
Este documento apresenta as respostas para a avaliação, com o raciocínio utilizado para a conclusão a respeito de cada alternativa das questões. Ao final, apresenta-se um apêndice com os códigos utilizados em cada questão e suas respectivas saídas.

\section{Questões e Respostas}
\subsection{Questão 1}
\textbf{Pergunta:} Ajuste um modelo de regressão de Weibull utilizando as seguintes covariáveis: \textit{yschool} - Anos de estudo; \textit{npartner} - Número de parceiros nos últimos 30 dias. Analise os resultados obtidos e classifique as alternativas. Obs: as covariáveis é incluída no parâmetro $\mu$.

\textbf{Resposta: D. Apenas II e III
}

\textit{I) Após o ajuste do modelo, identificou-se que as covariáveis yschool e npartner são significativas:} FALSA. Apenas a covariável \textit{yschool} é significativa segundo o teste com $\alpha = 5\%$ de significância, uma vez que $p-valor_{yschool} = 2.1e-06 < \alpha$ e $p-valor_{npartner} = 0.274 > \alpha$.

\textit{II) A estimativa obtida para a covariável yschool foi de 0.20882 com erro padrão de 0.04373. E a estimativa obtida para a covariável npartner foi de -0.07160 com erro padrão de 0.06548:} VERDADEIRA.


\textit{III) Após o ajuste do modelo, identificou-se que apenas a covariável yschool é significativa:} VERDADEIRA.

\textit{IV) Todas as estimativas obtidas neste modelo ajustado podem ser interpretadas:} FALSA. A covariável $\sigma$ não é interpretável, no sentido de que não se pode afirmar que aumentando-o em $n$ unidades terá um impacto de $p\%$ na variável resposta.


\subsection{Questão 2}
\textbf{Pergunta:} Ajuste um modelo de regressão de Weibull utilizando a covariável \textit{abdpain} - Presença de dor abdominal (1=sim e 0=não). Analise os resultados obtidos e selecione a alternativa correta. Obs: a covariável é incluida no parâmetro $\mu$.

\textbf{Resposta: C. VVFFV
} 

\textit{a) As estimativas obtidas da modelagem são $\mu_1 =  7.063$, $\mu_2 =-0.393$, $\sigma = 0.281$:} VERDADEIRA. Apesar de que o valor de $\sigma$ na verdade é negativo, considera-se esta afirmativa verdadeira devido a erro de digitação, na inexistência de outra alternativa correta para marcar.

\textit{b) A interpretação do parâmetro associado à variável abdpain é: à medida que a dor abdominal aumenta em uma unidade, espera-se que o tempo de sobrevivência do paciente diminua em um fator de $e^{-0.393} \approx 0.675$:} VERDADEIRA.


\textit{c) A matriz de variância-covariância é
\begin{equation}
C_{ov} = \begin{bmatrix}
0.007310457 & 0.0063106395 & -0.0016065125 \\
0.0063106395 & 0.0373636463 & -0.0003651316 \\
-0.0016065125 & -0.0003651316 & 0.0019946585
\end{bmatrix}
\end{equation}}

FALSA. A matriz de variância-covariância encontrada foi
\begin{equation}
C_{ov} = \begin{bmatrix}
0.007310457 & -0.0063106395 & -0.0016065125 \\
-0.0063106395 & 0.0373636463 & 0.0003651316 \\
-0.0016065125 & 0.0003651316 & 0.0019946585
\end{bmatrix}
\end{equation}


\textit{d) O parâmetro $\sigma$ representa o logaritmo do desvio padrão da distribuição Weibull:} FALSA. O parâmetro $\sigma$ é um dos parâmetros da função \textit{WEIrc}.


\textit{e) A estimativa do $\sigma$  indica que a forma da função de risco associada a esta distribuição está diminuindo ao longo do tempo:} VERDADEIRA. Na distribuição Weibull, quando o parâmetro de forma $\sigma$ assume valores negativos isso indica que a função de risco será uma função monotonicamente decrescente.


\subsection{Questão 3}
\textbf{Pergunta:} Ajuste o modelo de Cox e considere as cováriaveis: \textit{yschool} - Anos de estudo; npartner - Número de parceiros nos últimos 30 dias e \textit{abdpain} - Presença de dor abdominal (1=sim e 0=não). Interprete seus parâmetros acerca da razão de riscos e julgue as sentenças abaixo.

\textbf{Resposta: C. Apenas a afirmativa I é falsa}

\textit{I) O coeficiente negativo indica que uma diminuição no tempo de escolaridade está associado a uma redução no risco de reinfecção. O hazard ratio de 0.84945 significa que para cada unidade adicional no tempo de escolaridade, o risco de reinfecção diminui em aproximadamente $15\%$. Com um p-valor menor que 0.01, esta covariável é estatisticamente significativa, indicando que tempo de escolaridade é um fator relevante para o risco de reinfecção:} FALSA. O coeficiente negativo indica justamente uma relação inversa entre tempo de escolaridade e risco de reinfecção, ou seja, quanto menor o tempo de escolaridade, maior o risco de reinfecção.

\textit{II) O coeficiente positivo indica que um aumento no número de parceiros está associado a um aumento no risco de reinfecção. O hazard ratio de 1.05059 significa que para cada parceiro adicional, o risco de reinfecção aumenta em aproximadamente $5\%$. No entanto, esta covariável não é estatisticamente significativa no modelo, sugerindo que o número de parceiros pode não ter um impacto substancial no risco de reinfecção:} VERDADEIRA.


\textit{III) O coeficiente positivo indica que a presença de dor abdominal está associada a um aumento no risco de reinfecção. O hazard ratio de 1.34956 significa que a presença de dor abdominal aumenta o risco de reinfecção em aproximadamente $35\%$. Com um p-valor de 0.0411, esta covariável é estatisticamente significativa, indicando que a dor abdominal é um fator relevante para o risco de reinfecção:} VERDADEIRA.


\subsection{Questão 4}
\textbf{Pergunta:} Ajuste dois Modelos de Cox, primeiro considere as cováriaveis: \textit{npartner} e \textit{abdpain}. No segundo modelo adicione a interação entre as covariáveis \textit{“npartner”} e \textit{“abdpain”}. Acerca das afirmações abaixo classifique como verdadeira ou falsa.

\textbf{Resposta: A. FFVF}

\textit{I) Todas as covariáveis no primeiro modelo foram significativas a um nível de $10\%$ de significância mas ao considerar um nível de $5\%$ a covariável “abdpain” não é significativa:} FALSA. Apenas a covariável \textit{abdpain} é significativa para os dois níveis de significância sugeridos, uma vez que $p-valor_{abdpain} = 0.0466$ e $p-valor_{npartner} = 0.3766$.

\textit{II) O coeficiente estimado para a covariável “npartner” no primeiro modelo é positivo mas abaixo de 0.05, indicando que conforme o valor dela aumenta, o risco de falha diminui:} FALSA. Caso a covariável fosse significativa, a interpretação seria que, por ser positiva, a relação com o risco é proporcional, ou seja, conforme seu valor aumenta, o risco também aumenta.


\textit{III) A magnitude do coeficiente mostra a força da associação entre a covariável e o risco de falha. Coeficientes com maior valor absoluto têm um impacto mais significativo na função de risco. Com isso a covariável que indica dor abdominal é a que possui maior impacto no primeiro modelo de Cox ajustado:} VERDADEIRA.

\textit{IV) Ao ajustar o modelo que considera a interação entre as covariáveis “npartner” e “abdpain” verificamos que esta é a que possui menor p-valor, com isso é a mais significativa, portanto, sendo necessária considerá-la na modelagem final:} FALSA. De fato é a que possui menor $p-valor$. Entretanto, para todas as covariáveis o resultado indica que não são significativas, a ver: $p-valor_{npartner:abdpain} = 0.271$, $p-valor_{npartner} = 0.997$ e $p-valor_{abdpain} = 0.442$.



\subsection{Questão 5}
\textbf{Pergunta:} Ajuste dois modelos de cox, sendo o primeiro considerando apenas a covariável yschool e o segundo \textit{yschool} e \textit{npartner}. Indique qual das afimativas abaixo é verdadeira.

\textbf{Resposta: C. No segundo modelo, o coeficiente de yschool continua significativo, enquanto o coeficiente de npartner não é estatisticamente significativo.
}

\textit{a. A inclusão da covariável npartner no segundo modelo torna a covariável yschool significativa:} FALSA. A covariável \textit{yschool} já era significativa no primeiro modelo, tendo em vista que $p-valor_{yschool} = 1.09e-06$

\textit{b. O coeficiente de yschool no primeiro modelo indica que um aumento no tempo de escolaridade está associado a um aumento no risco de reinfecção:} FALSA. O coeficiente é negativo ($-0.16058$), o que indica uma relação inversamente proporcional, ou seja, o aumento no tempo de escolaridade está associado a uma diminuição no risco de reinfecção.

\textit{c. No segundo modelo, o coeficiente de yschool continua significativo, enquanto o coeficiente de npartner não é estatisticamente significativo:} VERDADEIRA.

\textit{d. A inclusão da covariável npartner no segundo modelo indica que um aumento no número de parceiros está associado a uma redução no risco de reinfecção para a cováriavel yschool, já que o coeficiente era -0.16047 e passou a ser -0.16161:} FALSA. A mudança no coeficiente de \textit{yschool} por si só não indica diretamente como \textit{npartner} está associado ao risco de reinfecção. Isso apenas mostra que o ajuste do modelo mudou ligeiramente o efeito estimado de \textit{yschool}.  A inclusão de uma nova covariável no modelo pode ajustar os efeitos das outras covariáveis devido à possível correlação entre elas, mas isso não implica diretamente em como uma covariável específica (\textit{npartner}) afeta o risco sem observar seu próprio coeficiente.

\onecolumn
\appendix
\section{Apêndice}
\subsection{Códigos e Saídas}
Esta seção contém os códigos utilizados para cada questão e suas respectivas saídas.

\subsubsection{Bibliotecas e Dados}

\begin{lstlisting}
install.packages("KMsurv")
install.packages("gamlss")
install.packages("gamlss.cens")
install.packages("broom")
install.packages("tidyverse")
install.packages("pammtools")
install.packages("numDeriv")
install.packages("KMsurv")

library(KMsurv)
library(gamlss)
library(gamlss.cens)
library(broom)
library(tidyverse)
library(pammtools)
library(numDeriv)
library(KMsurv)

data(std)
\end{lstlisting}


\subsubsection{Questão 1}

\begin{lstlisting}
gen.cens(WEI)
weibull_model1 = gamlss(Surv(time, rinfct) ~ yschool + npartner, family=WEIrc(mu.link = 'log', sigma.link = 'log'), data=std)
summary(weibull_model1)
\end{lstlisting}

\textbf{Saída:}
\begin{verbatim}
A censored family of distributions from WEI has been generated 
 and saved under the names:  
 dWEIrc pWEIrc qWEIrc WEIrc 
The type of censoring is right  
GAMLSS-RS iteration 1: Global Deviance = 5369.856 
GAMLSS-RS iteration 2: Global Deviance = 5369.772 
GAMLSS-RS iteration 3: Global Deviance = 5369.767 
GAMLSS-RS iteration 4: Global Deviance = 5369.766 
******************************************************************
Family:  c("WEIrc", "right censored Weibull") 

Call:  gamlss(formula = Surv(time, rinfct) ~ yschool + npartner,  
    family = WEIrc(mu.link = "log", sigma.link = "log"),      data = std) 

Fitting method: RS() 

------------------------------------------------------------------
Mu link function:  log
Mu Coefficients:
            Estimate Std. Error t value Pr(>|t|)    
(Intercept)  4.73257    0.49019   9.655  < 2e-16 ***
yschool      0.20881    0.04373   4.776  2.1e-06 ***
npartner    -0.07160    0.06548  -1.093    0.274    
---
Signif. codes:  0 ‘***’ 0.001 ‘**’ 0.01 ‘*’ 0.05 ‘.’ 0.1 ‘ ’ 1

------------------------------------------------------------------
Sigma link function:  log
Sigma Coefficients:
            Estimate Std. Error t value Pr(>|t|)    
(Intercept)  -0.2725     0.0446  -6.109  1.5e-09 ***
---
Signif. codes:  0 ‘***’ 0.001 ‘**’ 0.01 ‘*’ 0.05 ‘.’ 0.1 ‘ ’ 1

------------------------------------------------------------------
No. of observations in the fit:  877 
Degrees of Freedom for the fit:  4
      Residual Deg. of Freedom:  873 
                      at cycle:  4 
 
Global Deviance:     5369.766 
            AIC:     5377.766 
            SBC:     5396.872 
\end{verbatim}

\subsubsection{Questão 2}

\begin{lstlisting}
weibull_model2 = gamlss(Surv(time, rinfct) ~ abdpain, family=WEIrc(mu.link = 'log', sigma.link = 'log'), data=std)
summary(weibull_model2)

mcov2 = vcov(weibull_model2)
print(mcov2)
\end{lstlisting}

\textbf{Saída:}
\begin{verbatim}
GAMLSS-RS iteration 1: Global Deviance = 5390.44 
GAMLSS-RS iteration 2: Global Deviance = 5390.309 
GAMLSS-RS iteration 3: Global Deviance = 5390.301 
GAMLSS-RS iteration 4: Global Deviance = 5390.3 
******************************************************************
Family:  c("WEIrc", "right censored Weibull") 

Call:  
gamlss(formula = Surv(time, rinfct) ~ abdpain, family = WEIrc(mu.link = "log",  
    sigma.link = "log"), data = std) 

Fitting method: RS() 

------------------------------------------------------------------
Mu link function:  log
Mu Coefficients:
            Estimate Std. Error t value Pr(>|t|)    
(Intercept)   7.0634     0.0855  82.612   <2e-16 ***
abdpain      -0.3929     0.1933  -2.033   0.0424 *  
---
Signif. codes:  0 ‘***’ 0.001 ‘**’ 0.01 ‘*’ 0.05 ‘.’ 0.1 ‘ ’ 1

------------------------------------------------------------------
Sigma link function:  log
Sigma Coefficients:
            Estimate Std. Error t value Pr(>|t|)    
(Intercept) -0.28067    0.04466  -6.284 5.19e-10 ***
---
Signif. codes:  0 ‘***’ 0.001 ‘**’ 0.01 ‘*’ 0.05 ‘.’ 0.1 ‘ ’ 1

------------------------------------------------------------------
No. of observations in the fit:  877 
Degrees of Freedom for the fit:  3
      Residual Deg. of Freedom:  874 
                      at cycle:  4 
 
Global Deviance:     5390.3 
            AIC:     5396.3 
            SBC:     5410.63 

******************************************************************

             (Intercept)       abdpain   (Intercept)
(Intercept)  0.007310394 -0.0063105969 -0.0016064895
abdpain     -0.006310597  0.0373635227  0.0003651194
(Intercept) -0.001606490  0.0003651194  0.0019946522

\end{verbatim}


\subsubsection{Questão 3}

\begin{lstlisting}
cox_model1 = coxph(Surv(time, rinfct) ~ yschool + npartner + abdpain, data=std)
summary(cox_model1)
\end{lstlisting}

\textbf{Saída:}
\begin{verbatim}
Call:
coxph(formula = Surv(time, rinfct) ~ yschool + npartner + abdpain, 
    data = std)

  n= 877, number of events= 347 

             coef exp(coef) se(coef)      z Pr(>|z|)    
yschool  -0.16328   0.84935  0.03324 -4.912 9.02e-07 ***
npartner  0.04937   1.05061  0.04990  0.989   0.3225    
abdpain   0.30038   1.35038  0.14675  2.047   0.0407 *  
---
Signif. codes:  0 ‘***’ 0.001 ‘**’ 0.01 ‘*’ 0.05 ‘.’ 0.1 ‘ ’ 1

         exp(coef) exp(-coef) lower .95 upper .95
yschool     0.8494     1.1774    0.7958    0.9065
npartner    1.0506     0.9518    0.9527    1.1586
abdpain     1.3504     0.7405    1.0128    1.8004

Concordance= 0.575  (se = 0.017 )
Likelihood ratio test= 29.02  on 3 df,   p=2e-06
Wald test            = 28.73  on 3 df,   p=3e-06
Score (logrank) test = 28.58  on 3 df,   p=3e-06
\end{verbatim}


\subsubsection{Questão 4}

\begin{lstlisting}
cox_model2 = coxph(Surv(time, rinfct) ~ npartner + abdpain, data=std)
summary(cox_model2)

cox_model3 = coxph(Surv(time, rinfct) ~ npartner * abdpain, data=std)
summary(cox_model3)
\end{lstlisting}

\textbf{Saída:}
\begin{verbatim}
Call:
coxph(formula = Surv(time, rinfct) ~ npartner + abdpain, data = std)

  n= 877, number of events= 347 

            coef exp(coef) se(coef)     z Pr(>|z|)  
npartner 0.04343   1.04438  0.04911 0.884   0.3766  
abdpain  0.29196   1.33905  0.14671 1.990   0.0466 *
---
Signif. codes:  0 ‘***’ 0.001 ‘**’ 0.01 ‘*’ 0.05 ‘.’ 0.1 ‘ ’ 1

         exp(coef) exp(-coef) lower .95 upper .95
npartner     1.044     0.9575    0.9485     1.150
abdpain      1.339     0.7468    1.0044     1.785

Concordance= 0.519  (se = 0.016 )
Likelihood ratio test= 4.66  on 2 df,   p=0.1
Wald test            = 5.04  on 2 df,   p=0.08
Score (logrank) test = 5.06  on 2 df,   p=0.08

******************************************************************

Call:
coxph(formula = Surv(time, rinfct) ~ npartner * abdpain, data = std)

  n= 877, number of events= 347 

                      coef exp(coef)  se(coef)     z Pr(>|z|)
npartner         0.0002588 1.0002588 0.0660843 0.004    0.997
abdpain          0.1506696 1.1626124 0.1959681 0.769    0.442
npartner:abdpain 0.0990160 1.1040840 0.0899701 1.101    0.271

                 exp(coef) exp(-coef) lower .95 upper .95
npartner             1.000     0.9997    0.8787     1.139
abdpain              1.163     0.8601    0.7918     1.707
npartner:abdpain     1.104     0.9057    0.9256     1.317

Concordance= 0.519  (se = 0.016 )
Likelihood ratio test= 5.79  on 3 df,   p=0.1
Wald test            = 7.33  on 3 df,   p=0.06
Score (logrank) test = 7.76  on 3 df,   p=0.05


\end{verbatim}

\subsubsection{Questão 5}

\begin{lstlisting}
cox_model4 = coxph(Surv(time, rinfct) ~ yschool, data=std)
summary(cox_model4)

cox_model5 = coxph(Surv(time, rinfct) ~ yschool + npartner, data=std)
summary(cox_model5)
\end{lstlisting}

\textbf{Saída:}
\begin{verbatim}
Call:
coxph(formula = Surv(time, rinfct) ~ yschool, data = std)

  n= 877, number of events= 347 

            coef exp(coef) se(coef)      z Pr(>|z|)    
yschool -0.16058   0.85165  0.03294 -4.874 1.09e-06 ***
---
Signif. codes:  0 ‘***’ 0.001 ‘**’ 0.01 ‘*’ 0.05 ‘.’ 0.1 ‘ ’ 1

        exp(coef) exp(-coef) lower .95 upper .95
yschool    0.8517      1.174    0.7984    0.9085

Concordance= 0.567  (se = 0.016 )
Likelihood ratio test= 23.95  on 1 df,   p=1e-06
Wald test            = 23.76  on 1 df,   p=1e-06
Score (logrank) test = 23.5  on 1 df,   p=1e-06

******************************************************************

Call:
coxph(formula = Surv(time, rinfct) ~ yschool + npartner, data = std)

  n= 877, number of events= 347 

             coef exp(coef) se(coef)      z Pr(>|z|)    
yschool  -0.16172   0.85068  0.03305 -4.893 9.95e-07 ***
npartner  0.05650   1.05813  0.04987  1.133    0.257    
---
Signif. codes:  0 ‘***’ 0.001 ‘**’ 0.01 ‘*’ 0.05 ‘.’ 0.1 ‘ ’ 1

         exp(coef) exp(-coef) lower .95 upper .95
yschool     0.8507     1.1755    0.7973    0.9076
npartner    1.0581     0.9451    0.9596    1.1668

Concordance= 0.565  (se = 0.017 )
Likelihood ratio test= 25.1  on 2 df,   p=4e-06
Wald test            = 24.83  on 2 df,   p=4e-06
Score (logrank) test = 24.6  on 2 df,   p=5e-06
\end{verbatim}


\end{document}
